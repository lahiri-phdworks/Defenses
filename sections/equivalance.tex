\begin{frame}{Verification for Program Equivalence.}
    Program equivalence requires that, any two executions of a pair of programs, $\prog_1$ and $\prog_2$ on same the same input $\vec{x}$, must yield the same outputs.
    \pause
    \begin{tcolorbox}[
        colback=white,
        colframe=green,
        colbacktitle=white!70!green,
        coltitle=black,
        title=\textbf{A Verification Problem},
        enhanced,
        attach boxed title to top left={yshift=-2mm, xshift=0.5cm},%
        ]
        \[
        \forall \vec{x}.\ \prog_1(\vec{x}) = \prog_2(\vec{x})
        \]
    \end{tcolorbox}
    %\pause
    %\begin{tcolorbox}[
    %    colback=white,
    %    colframe=red,
    %    colbacktitle=white!70!red,
    %    coltitle=black,
    %    title=\textbf{Harder Problem!},
    %    enhanced,
    %    attach boxed title to top left={yshift=-2mm, xshift=0.5cm},%
    %]
    %    \[Finding \prog_1\ or\ \prog_2\ may\ only\ be\ \textit{partially specified}.\ \]
    %\end{tcolorbox}
\end{frame}

\begin{frame}{Sketching for Program Equivalence.}
    Given a reference program $\prog_1$ and a \textit{partial} program $\prog_2$ that has a hole $\boxed{\cdot}$, we are interested in \textit{completing} the partial program by synthesizing an expression to fill the hole such that the two programs are rendered semantically equivalent.
    \pause
    \begin{tcolorbox}[
        colback=white,
        colframe=blue,
        colbacktitle=white!70!blue,
        coltitle=black,
        title=\textbf{Formal Definition},
        enhanced,
        attach boxed title to top left={yshift=-2mm, xshift=0.5cm},%
        ]
        \[ \exists \E \in \lang.\ \forall \vec{x}.\ \prog_1(\vec{x}) = \proghole{2}{\E}(\vec{x}) \]
    \end{tcolorbox}
\end{frame}

\begin{frame}[fragile]{Sketching for Program Equivalence}
    %    \frametitle{Program Equivalence}
    \begin{figure}[t]
        \begin{subfigure}{0.48\textwidth}
            \begin{minted}{c}
                int |$\prog_1$|(int n){
                    |\colorbox{yellow!20!white}{assume(n > 1);}|
                    int i = 0, ans = 0;
                    while(i < (n - 1)){
                        i = i + 1;
                        ans = ans + (5 * i) + 1;
                    }
                    return ans + 1;
                }
            \end{minted}
            \caption{Program 1 \label{list:p1}}
        \end{subfigure}
        \begin{subfigure}{0.48\textwidth}
            \begin{minted}[tabsize=2, numbersep=0em, tabsize=1, fontsize=\small,  highlightlines={15}, escapeinside=||]{c}
                int |$\proghole{2}{.}$|(int n){
                    |\colorbox{yellow!20!white}{assume(n > 1);}|
                    int x = 0, y = 0, z = n;
                    while(z |$\neq$| 0){
                        z = z - 1;
                        x = |\hole{\cdot}|;
                        y = y + 1;
                    }
                    return x + y;
                }
            \end{minted}
            \caption{Program 2 (with hole \hole{\cdot})\label{list:p2}}
        \end{subfigure}
        %        \caption{Two programs $\prog_1$ and $\proghole{2}{.}$ for program equivalence.\label{list:motiv:eq}}
    \end{figure}
    \pause
    \begin{tcolorbox}[
        colback=white,
        colframe=blue,
        colbacktitle=white!70!blue,
        coltitle=black,
        title=\textbf{Post condition for sketching.},
        enhanced,
        attach boxed title to top left={yshift=-2mm, xshift=0.5cm},%
        ]
        \[ \exists \E \in \lang.\ \forall \vec{x}.\ \prog_1(\vec{x}) = \proghole{2}{\E}(\vec{x}) \]
        \pause
        \vspace{-10pt}
        \[
        \colorbox{green!20!white}{$assert(ans + 1 = x + y)$}
        \]
    \end{tcolorbox}
\end{frame}

\begin{frame}[fragile]{Sketching for Program Equivalence}
    %    \frametitle{Program Equivalence}
    \begin{figure}[t]
        \begin{subfigure}{0.48\textwidth}
            \begin{minted}{c}
                int |$\prog_1$|(int n){
                    |\colorbox{yellow!20!white}{assume(n > 1);}|
                    int i = 0, ans = 0;
                    while(i < (n - 1)){
                        i = i + 1;
                        ans = ans + (5 * i) + 1;
                    }
                    return ans + 1;
                }
            \end{minted}
            \caption{Program 1 \label{list:p1}}
        \end{subfigure}
        \begin{subfigure}{0.48\textwidth}
            \begin{minted}[tabsize=2, numbersep=0em, tabsize=1, fontsize=\small,  highlightlines={15}, escapeinside=||]{c}
                int |$\proghole{2}{.}$|(int n){
                    |\colorbox{yellow!20!white}{assume(n > 1);}|
                    int x = 0, y = 0, z = n;
                    while(z |$\neq$| 0){
                        z = z - 1;
                        x = |\hole{x + 6 \cdot y - n}|;
                        y = y + 1;
                    }
                    return x + y;
                }
            \end{minted}
            \caption{Program 2 (with hole \hole{\cdot})\label{list:p2}}
        \end{subfigure}
        %        \caption{Two programs $\prog_1$ and $\proghole{2}{.}$ for program equivalence.\label{list:motiv:eq}}
    \end{figure}
\end{frame}
