\begin{frame}{Relational property with (semantic) quantitative objectives.}

\end{frame}

\begin{frame}{All Gammas!}
    \textbf{Monotonicity, Robustness:} A preference on a completion could be the one that minimizes the distance between any two responses of the program.
    \begin{align*}
        \Gamma(\proghole{}{\E}(\vec{x_1}), \proghole{}{\E}(\vec{x_2})) \dfn ||\proghole{}{\E}(\vec{x_1}) - \proghole{}{\E}(\vec{x_2})||
    \end{align*}
    \textbf{Weak Non-Interference, Weak Equivalence:} A preference on a completion could be the one that minimizes the distance between any two responses of the program.
    \begin{align*}
        \Gamma(\proghole{}{\E}(\vec{x_1}), \proghole{}{\E}(\vec{x_2})) \dfn ||\proghole{}{\E}(\vec{x_1}) - \proghole{}{\E}(\vec{x_2})||
    \end{align*}
    \textbf{Group Fairness:} One may design many preference metrics over completions. One metric could be to prefer completions where the deviation in responses between candidates of two populations is small for similar candidates.
    \begin{align*}
        \Gamma(\proghole{}{\E}(\vec{x_1}, {s_1}), \proghole{}{\E}(\vec{x_2}, {s_2})) \dfn \textrm{if } (s_1 < s_2 \land \vec{x_1} \sim \vec{x_2})\\ \textrm{ then }||\proghole{}{\E}(\vec{x_1}, {y_1}) - \proghole{}{\E}(\vec{x_2}, {y_2})|| \textrm{ else } 0
    \end{align*}
\end{frame}
